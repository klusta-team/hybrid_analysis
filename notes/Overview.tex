


% Header, overrides base

    % Make sure that the sphinx doc style knows who it inherits from.
    \def\sphinxdocclass{article}

    % Declare the document class
    \documentclass[letterpaper,10pt,english]{/usr/local/lib/python2.7/dist-packages/Sphinx-1.2b3-py2.7.egg/sphinx/texinputs/sphinxhowto}

    % Imports
    \usepackage[utf8]{inputenc}
    \DeclareUnicodeCharacter{00A0}{\\nobreakspace}
    \usepackage[T1]{fontenc}
    \usepackage{babel}
    \usepackage{times}
    \usepackage{import}
    \usepackage[Bjarne]{/usr/local/lib/python2.7/dist-packages/Sphinx-1.2b3-py2.7.egg/sphinx/texinputs/fncychap}
    \usepackage{longtable}
    \usepackage{/usr/local/lib/python2.7/dist-packages/Sphinx-1.2b3-py2.7.egg/sphinx/texinputs/sphinx}
    \usepackage{multirow}

    \usepackage{amsmath}
    \usepackage{amssymb}
    \usepackage{ucs}
    \usepackage{enumerate}

    % Used to make the Input/Output rules follow around the contents.
    \usepackage{needspace}

    % Pygments requirements
    \usepackage{fancyvrb}
    \usepackage{color}
    % ansi colors additions
    \definecolor{darkgreen}{rgb}{.12,.54,.11}
    \definecolor{lightgray}{gray}{.95}
    \definecolor{brown}{rgb}{0.54,0.27,0.07}
    \definecolor{purple}{rgb}{0.5,0.0,0.5}
    \definecolor{darkgray}{gray}{0.25}
    \definecolor{lightred}{rgb}{1.0,0.39,0.28}
    \definecolor{lightgreen}{rgb}{0.48,0.99,0.0}
    \definecolor{lightblue}{rgb}{0.53,0.81,0.92}
    \definecolor{lightpurple}{rgb}{0.87,0.63,0.87}
    \definecolor{lightcyan}{rgb}{0.5,1.0,0.83}

    % Needed to box output/input
    \usepackage{tikz}
        \usetikzlibrary{calc,arrows,shadows}
    \usepackage[framemethod=tikz]{mdframed}

    \usepackage{alltt}

    % Used to load and display graphics
    \usepackage{graphicx}
    \graphicspath{ {figs/} }
    \usepackage[Export]{adjustbox} % To resize

    % used so that images for notebooks which have spaces in the name can still be included
    \usepackage{grffile}


    % For formatting output while also word wrapping.
    \usepackage{listings}
    \lstset{breaklines=true}
    \lstset{basicstyle=\small\ttfamily}
    \def\smaller{\fontsize{9.5pt}{9.5pt}\selectfont}

    %Pygments definitions
    
\makeatletter
\def\PY@reset{\let\PY@it=\relax \let\PY@bf=\relax%
    \let\PY@ul=\relax \let\PY@tc=\relax%
    \let\PY@bc=\relax \let\PY@ff=\relax}
\def\PY@tok#1{\csname PY@tok@#1\endcsname}
\def\PY@toks#1+{\ifx\relax#1\empty\else%
    \PY@tok{#1}\expandafter\PY@toks\fi}
\def\PY@do#1{\PY@bc{\PY@tc{\PY@ul{%
    \PY@it{\PY@bf{\PY@ff{#1}}}}}}}
\def\PY#1#2{\PY@reset\PY@toks#1+\relax+\PY@do{#2}}

\expandafter\def\csname PY@tok@gd\endcsname{\def\PY@tc##1{\textcolor[rgb]{0.63,0.00,0.00}{##1}}}
\expandafter\def\csname PY@tok@gu\endcsname{\let\PY@bf=\textbf\def\PY@tc##1{\textcolor[rgb]{0.50,0.00,0.50}{##1}}}
\expandafter\def\csname PY@tok@gt\endcsname{\def\PY@tc##1{\textcolor[rgb]{0.00,0.25,0.82}{##1}}}
\expandafter\def\csname PY@tok@gs\endcsname{\let\PY@bf=\textbf}
\expandafter\def\csname PY@tok@gr\endcsname{\def\PY@tc##1{\textcolor[rgb]{1.00,0.00,0.00}{##1}}}
\expandafter\def\csname PY@tok@cm\endcsname{\let\PY@it=\textit\def\PY@tc##1{\textcolor[rgb]{0.25,0.50,0.50}{##1}}}
\expandafter\def\csname PY@tok@vg\endcsname{\def\PY@tc##1{\textcolor[rgb]{0.10,0.09,0.49}{##1}}}
\expandafter\def\csname PY@tok@m\endcsname{\def\PY@tc##1{\textcolor[rgb]{0.40,0.40,0.40}{##1}}}
\expandafter\def\csname PY@tok@mh\endcsname{\def\PY@tc##1{\textcolor[rgb]{0.40,0.40,0.40}{##1}}}
\expandafter\def\csname PY@tok@go\endcsname{\def\PY@tc##1{\textcolor[rgb]{0.50,0.50,0.50}{##1}}}
\expandafter\def\csname PY@tok@ge\endcsname{\let\PY@it=\textit}
\expandafter\def\csname PY@tok@vc\endcsname{\def\PY@tc##1{\textcolor[rgb]{0.10,0.09,0.49}{##1}}}
\expandafter\def\csname PY@tok@il\endcsname{\def\PY@tc##1{\textcolor[rgb]{0.40,0.40,0.40}{##1}}}
\expandafter\def\csname PY@tok@cs\endcsname{\let\PY@it=\textit\def\PY@tc##1{\textcolor[rgb]{0.25,0.50,0.50}{##1}}}
\expandafter\def\csname PY@tok@cp\endcsname{\def\PY@tc##1{\textcolor[rgb]{0.74,0.48,0.00}{##1}}}
\expandafter\def\csname PY@tok@gi\endcsname{\def\PY@tc##1{\textcolor[rgb]{0.00,0.63,0.00}{##1}}}
\expandafter\def\csname PY@tok@gh\endcsname{\let\PY@bf=\textbf\def\PY@tc##1{\textcolor[rgb]{0.00,0.00,0.50}{##1}}}
\expandafter\def\csname PY@tok@ni\endcsname{\let\PY@bf=\textbf\def\PY@tc##1{\textcolor[rgb]{0.60,0.60,0.60}{##1}}}
\expandafter\def\csname PY@tok@nl\endcsname{\def\PY@tc##1{\textcolor[rgb]{0.63,0.63,0.00}{##1}}}
\expandafter\def\csname PY@tok@nn\endcsname{\let\PY@bf=\textbf\def\PY@tc##1{\textcolor[rgb]{0.00,0.00,1.00}{##1}}}
\expandafter\def\csname PY@tok@no\endcsname{\def\PY@tc##1{\textcolor[rgb]{0.53,0.00,0.00}{##1}}}
\expandafter\def\csname PY@tok@na\endcsname{\def\PY@tc##1{\textcolor[rgb]{0.49,0.56,0.16}{##1}}}
\expandafter\def\csname PY@tok@nb\endcsname{\def\PY@tc##1{\textcolor[rgb]{0.00,0.50,0.00}{##1}}}
\expandafter\def\csname PY@tok@nc\endcsname{\let\PY@bf=\textbf\def\PY@tc##1{\textcolor[rgb]{0.00,0.00,1.00}{##1}}}
\expandafter\def\csname PY@tok@nd\endcsname{\def\PY@tc##1{\textcolor[rgb]{0.67,0.13,1.00}{##1}}}
\expandafter\def\csname PY@tok@ne\endcsname{\let\PY@bf=\textbf\def\PY@tc##1{\textcolor[rgb]{0.82,0.25,0.23}{##1}}}
\expandafter\def\csname PY@tok@nf\endcsname{\def\PY@tc##1{\textcolor[rgb]{0.00,0.00,1.00}{##1}}}
\expandafter\def\csname PY@tok@si\endcsname{\let\PY@bf=\textbf\def\PY@tc##1{\textcolor[rgb]{0.73,0.40,0.53}{##1}}}
\expandafter\def\csname PY@tok@s2\endcsname{\def\PY@tc##1{\textcolor[rgb]{0.73,0.13,0.13}{##1}}}
\expandafter\def\csname PY@tok@vi\endcsname{\def\PY@tc##1{\textcolor[rgb]{0.10,0.09,0.49}{##1}}}
\expandafter\def\csname PY@tok@nt\endcsname{\let\PY@bf=\textbf\def\PY@tc##1{\textcolor[rgb]{0.00,0.50,0.00}{##1}}}
\expandafter\def\csname PY@tok@nv\endcsname{\def\PY@tc##1{\textcolor[rgb]{0.10,0.09,0.49}{##1}}}
\expandafter\def\csname PY@tok@s1\endcsname{\def\PY@tc##1{\textcolor[rgb]{0.73,0.13,0.13}{##1}}}
\expandafter\def\csname PY@tok@sh\endcsname{\def\PY@tc##1{\textcolor[rgb]{0.73,0.13,0.13}{##1}}}
\expandafter\def\csname PY@tok@sc\endcsname{\def\PY@tc##1{\textcolor[rgb]{0.73,0.13,0.13}{##1}}}
\expandafter\def\csname PY@tok@sx\endcsname{\def\PY@tc##1{\textcolor[rgb]{0.00,0.50,0.00}{##1}}}
\expandafter\def\csname PY@tok@bp\endcsname{\def\PY@tc##1{\textcolor[rgb]{0.00,0.50,0.00}{##1}}}
\expandafter\def\csname PY@tok@c1\endcsname{\let\PY@it=\textit\def\PY@tc##1{\textcolor[rgb]{0.25,0.50,0.50}{##1}}}
\expandafter\def\csname PY@tok@kc\endcsname{\let\PY@bf=\textbf\def\PY@tc##1{\textcolor[rgb]{0.00,0.50,0.00}{##1}}}
\expandafter\def\csname PY@tok@c\endcsname{\let\PY@it=\textit\def\PY@tc##1{\textcolor[rgb]{0.25,0.50,0.50}{##1}}}
\expandafter\def\csname PY@tok@mf\endcsname{\def\PY@tc##1{\textcolor[rgb]{0.40,0.40,0.40}{##1}}}
\expandafter\def\csname PY@tok@err\endcsname{\def\PY@bc##1{\setlength{\fboxsep}{0pt}\fcolorbox[rgb]{1.00,0.00,0.00}{1,1,1}{\strut ##1}}}
\expandafter\def\csname PY@tok@kd\endcsname{\let\PY@bf=\textbf\def\PY@tc##1{\textcolor[rgb]{0.00,0.50,0.00}{##1}}}
\expandafter\def\csname PY@tok@ss\endcsname{\def\PY@tc##1{\textcolor[rgb]{0.10,0.09,0.49}{##1}}}
\expandafter\def\csname PY@tok@sr\endcsname{\def\PY@tc##1{\textcolor[rgb]{0.73,0.40,0.53}{##1}}}
\expandafter\def\csname PY@tok@mo\endcsname{\def\PY@tc##1{\textcolor[rgb]{0.40,0.40,0.40}{##1}}}
\expandafter\def\csname PY@tok@kn\endcsname{\let\PY@bf=\textbf\def\PY@tc##1{\textcolor[rgb]{0.00,0.50,0.00}{##1}}}
\expandafter\def\csname PY@tok@mi\endcsname{\def\PY@tc##1{\textcolor[rgb]{0.40,0.40,0.40}{##1}}}
\expandafter\def\csname PY@tok@gp\endcsname{\let\PY@bf=\textbf\def\PY@tc##1{\textcolor[rgb]{0.00,0.00,0.50}{##1}}}
\expandafter\def\csname PY@tok@o\endcsname{\def\PY@tc##1{\textcolor[rgb]{0.40,0.40,0.40}{##1}}}
\expandafter\def\csname PY@tok@kr\endcsname{\let\PY@bf=\textbf\def\PY@tc##1{\textcolor[rgb]{0.00,0.50,0.00}{##1}}}
\expandafter\def\csname PY@tok@s\endcsname{\def\PY@tc##1{\textcolor[rgb]{0.73,0.13,0.13}{##1}}}
\expandafter\def\csname PY@tok@kp\endcsname{\def\PY@tc##1{\textcolor[rgb]{0.00,0.50,0.00}{##1}}}
\expandafter\def\csname PY@tok@w\endcsname{\def\PY@tc##1{\textcolor[rgb]{0.73,0.73,0.73}{##1}}}
\expandafter\def\csname PY@tok@kt\endcsname{\def\PY@tc##1{\textcolor[rgb]{0.69,0.00,0.25}{##1}}}
\expandafter\def\csname PY@tok@ow\endcsname{\let\PY@bf=\textbf\def\PY@tc##1{\textcolor[rgb]{0.67,0.13,1.00}{##1}}}
\expandafter\def\csname PY@tok@sb\endcsname{\def\PY@tc##1{\textcolor[rgb]{0.73,0.13,0.13}{##1}}}
\expandafter\def\csname PY@tok@k\endcsname{\let\PY@bf=\textbf\def\PY@tc##1{\textcolor[rgb]{0.00,0.50,0.00}{##1}}}
\expandafter\def\csname PY@tok@se\endcsname{\let\PY@bf=\textbf\def\PY@tc##1{\textcolor[rgb]{0.73,0.40,0.13}{##1}}}
\expandafter\def\csname PY@tok@sd\endcsname{\let\PY@it=\textit\def\PY@tc##1{\textcolor[rgb]{0.73,0.13,0.13}{##1}}}

\def\PYZbs{\char`\\}
\def\PYZus{\char`\_}
\def\PYZob{\char`\{}
\def\PYZcb{\char`\}}
\def\PYZca{\char`\^}
\def\PYZam{\char`\&}
\def\PYZlt{\char`\<}
\def\PYZgt{\char`\>}
\def\PYZsh{\char`\#}
\def\PYZpc{\char`\%}
\def\PYZdl{\char`\$}
\def\PYZti{\char`\~}
% for compatibility with earlier versions
\def\PYZat{@}
\def\PYZlb{[}
\def\PYZrb{]}
\makeatother


    %Set pygments styles if needed...
    
        \definecolor{nbframe-border}{rgb}{0.867,0.867,0.867}
        \definecolor{nbframe-bg}{rgb}{0.969,0.969,0.969}
        \definecolor{nbframe-in-prompt}{rgb}{0.0,0.0,0.502}
        \definecolor{nbframe-out-prompt}{rgb}{0.545,0.0,0.0}

        \newenvironment{ColorVerbatim}
        {\begin{mdframed}[%
            roundcorner=1.0pt, %
            backgroundcolor=nbframe-bg, %
            userdefinedwidth=1\linewidth, %
            leftmargin=0.1\linewidth, %
            innerleftmargin=0pt, %
            innerrightmargin=0pt, %
            linecolor=nbframe-border, %
            linewidth=1pt, %
            usetwoside=false, %
            everyline=true, %
            innerlinewidth=3pt, %
            innerlinecolor=nbframe-bg, %
            middlelinewidth=1pt, %
            middlelinecolor=nbframe-bg, %
            outerlinewidth=0.5pt, %
            outerlinecolor=nbframe-border, %
            needspace=0pt
        ]}
        {\end{mdframed}}
        
        \newenvironment{InvisibleVerbatim}
        {\begin{mdframed}[leftmargin=0.1\linewidth,innerleftmargin=3pt,innerrightmargin=3pt, userdefinedwidth=1\linewidth, linewidth=0pt, linecolor=white, usetwoside=false]}
        {\end{mdframed}}

        \renewenvironment{Verbatim}[1][\unskip]
        {\begin{alltt}\smaller}
        {\end{alltt}}
    

    % Help prevent overflowing lines due to urls and other hard-to-break 
    % entities.  This doesn't catch everything...
    \sloppy

    % Document level variables
    \title{Overview}
    \date{November 22, 2013}
    \release{}
    \author{Unknown Author}
    \renewcommand{\releasename}{}

    % TODO: Add option for the user to specify a logo for his/her export.
    \newcommand{\sphinxlogo}{}

    % Make the index page of the document.
    \makeindex

    % Import sphinx document type specifics.
     


% Body

    % Start of the document
    \begin{document}

        
            \maketitle
        

        


        
        \section{Important Preliminaries:}

1). The .kwik format has to be modified so that it can store more than
one clustering.

2). Which hashing algorithm? Something from
\href{http://docs.python.org/2/library/hashlib.html}{http://docs.python.org/2/library/hashlib.html}.
We obviously don't need cryptography.

\section{Original data files}

There is a folder consisting of the three recordings (.dat files) of
Mariano Belluscio. One .dat file has been clustered
(\texttt{n6mab031109.dat}), with method \texttt{MKKdistfloat} and has a
number of nice clusters out of which we can choose a few suitable donor
cells. There are two other recordings \texttt{n6mab041109.dat} and
\texttt{n6mab061109.dat}; these will act as ``acceptor'' datasets.

\section{Hybrid dataset creation}

Minimal information necessary for defining a hybrid dataset is a 4-tuple
of the form: \texttt{(acceptor, donor\_id, amplitude, time\_series)},
where \texttt{acceptor} is the dataset which receives spikes from
another analogously recorded dataset \texttt{donor} at the times
specified by \texttt{time\_series}. Each element of the 4-tuple may be
generated by a range of other parameters, e.g.

\begin{enumerate}[1)]
\item
  A rate, \texttt{r} could define a \texttt{time\_series} consisting of
  regular spiking at \texttt{r} Hz. This could be created by a function:

  def: create\_regular\_resfile(rate, sampling\_rate, starttime,
  endtime, resname): `''Creates a regular .res file with firing rate r
  Hz samples'''
\item
  The donor\_id could be derived from information, e.g.~the 3-tuple
  (donor,donorcluid,donorcluster) pertaining to the donor dataset
  (donor), the detection method and clustering algorithm (donorcluid)
  and cluster (donorcluster) number of the cell that is added to the
  acceptor dataset.

  def: create\_donor\_id(donor,donorcluid,donorcluster): '''Outputs
  donor identity files:

  donor\_id = donor\_donorcluid\_donorcluster, (which typically looks
  like: n6mab031109\_MKKdistfloat\_54`).

  meanspike\_file\_id = a file called donor\_id.msua.1.

  meanmask\_file = a file called donor\_id.amsk.1.`''
\end{enumerate}

We can have a folder containing donor\_ids, or a list of 3-tuples
generating to donor\_ids.

We can create a hybrid dataset
\texttt{D(acceptor, donor\_id, amplitude, time\_series)} (which we shall
abbreviate to \texttt{D} when there is no ambiguity) using the Python
function:

\begin{verbatim}
def: create_hybrid_datfile(acceptor, donor_id, amplitude, time_series):

'''This function outputs a raw datafile called, 
      Hash(D).dat, in the folder Hash(D)'''

def: convert_to_kwik():
        '''Converts Hash(D).dat to Hash(D).kwd and Hash(D).kwik'''
\end{verbatim}

, where Hash(var) is the hash produced from the variables \texttt{var}
via some hash function.

Question: The raw waveform from one recording on non-relevant channels
could contain wildly inappropriate waveforms. I got around this before
by only added parts of the wave form corresponding to the .amsk.1 file.

NOTE: \texttt{D} implicitly contains the groundtruth times (equivalent
to a .res and a .clu file) for the hybrid spike times and hybrid
clusters.

\section{Running SpikeDetekt with various parameters}

We aim to find the optimal detection strategy, but running spikedetekt
on Hash(D).dat several times using a variety of parameters,
\texttt{params}, which we shall denote \texttt{p}, resulting in two
files in the folder

\texttt{Hash(D)\_Hash(p)}:

\texttt{Hash(D)\_Hash(p).kwx}

\texttt{Hash(D)\_Hash(p).kwik}

The feature and mask vectors, corresponding to the .fet and .fmask files
are stored in the .kwx file, these will later be required by Masked
KlustaKwik.

The parameters are stored in a global Python dictionary of variables
which can be accessed by all modules of SpikeDetekt. The default value
of these parameters are stored in
\texttt{spikedetekt/spikedetekt/defaultparameters.py}.

In the first phase we would like to finish testing the effect of varying
certain parameters on the quality of spike detection and alignment,
e.g.~testing two-threshold detection by keeping the upper threshold
constant and varying the lower threshold. We can vary one parameter and
keep the others constant by specifying custom default parameters in the
file \texttt{usualparameters.py} . We will specify a 2D detection
window, \texttt{Sigma} consisting of a minimal time jitter window and a
threshold of minimal mask similarity measure (\textgreater{}0.8); a
spike will be denoted as ``successfully detected'' if it lies within
this detection window. This will be implemented in some Python functions
in \texttt{evaluate\_detection.py}. This will have two outputs:

\begin{enumerate}[1)]
\item
  \texttt{detection\_statistics(D, p,Sigma)} will be a set of scripts
  which will measure the efficacy of the detection \texttt{p}. e.g.~we
  could have a function: def: test\_detection\_algorithm(D,p,Sigma):
  `''Test spike detection algorithm on a hybrid dataset'''
\item
  A \texttt{derived groundtruth(D, p,Sigma)} relative to the detection
  (i.e.~equivalent to a .clu file which is commensurate with the output
  .res file of detected spikes(those found in the window specified by
  \texttt{Sigma}, which specifies which spikes are background (in
  cluster 0) and which are hybrid (in clusters 1, 2, \ldots{}
  num\_hybrids. This clustering can be stored in
  Hash(D)\emph{Hash(p)}Sigma.kwik.
\end{enumerate}

\section{Supervised Learning}

To perform supervised learning in the form of a support vector machine
(SVM) on the groundtruth obtained after running SpikeDetekt and applying
detection criterion \texttt{Sigma}, we use the Python machine learning
package, \texttt{sklearn}. We will run SVM with a different kernels and
their associated parameters, and a grid of class weights. We shall
denote these parameters, s. The set of functions
\texttt{supervised\_learning} should output a set of clusterings from
which we obtain an ROC curve, providing an upper bound for performance
of any unsupervised algorithm. This set of clusterings will be stored in
the file \texttt{(D, p,Sigma,s)}.kwik.

\section{Clustering using Masked KlustaKwik}

Given detection, extraction of features and masking, specified in the
files \texttt{Hash(D)\_Hash(p).kwx}, we can now run KlustaKwik on the
.fet and .fmask files obtained from \texttt{Hash(D)\_Hash(p).kwx}. with
a set of parameters which we shall denote \texttt{k} which will be
defined a dictionary which will output a bash script for running
KlustaKwik,

\begin{verbatim}
def: get_KK_input(D,p,k):
'''Will output the .fet and .fmask files in a folder, CartesianHash_(D,p,k)'''

def: make_KK_runscripts(D,p,k):
'''Produces bash scripts pertaining to parameters k that run KlustaKwik on (D,p) '''

def: make_bash_hash(D,p,k):
'''Outputs the hash of (D,p,k) and the bash script'''
\end{verbatim}

This will result in a clustering which will be stored in the file
\texttt{Hash(D)\_Hash(p)\_Hash(k).kwx}. An analysis script can be run on
\texttt{Hash(D)\_Hash(p)\_Hash(k).kwx} and the associated groundtruth,
\texttt{Hash(D)\_Hash(p)\_Sigma.kwik} to obtain the confusion matrix.
This will be stored as a pickle or a textfile
\texttt{Hash(D)\_Hash(p)\_Hash(k).p}. M\# \#Supercomputer Scripts will
be required for sending jobs to Legion. These jobs will usually be SVM
and Masked KlustaKwik.

    % Make sure that atleast 4 lines are below the HR
    \needspace{4\baselineskip}

    
        \vspace{6pt}
        \makebox[0.1\linewidth]{\smaller\hfill\tt\color{nbframe-in-prompt}In\hspace{4pt}{[}{]}:\hspace{4pt}}\\*
        \vspace{-2.65\baselineskip}
        \begin{ColorVerbatim}
            \vspace{-0.7\baselineskip}
            \begin{Verbatim}[commandchars=\\\{\}]

\end{Verbatim}

            
                \vspace{-0.2\baselineskip}
            
        \end{ColorVerbatim}
    


    % Make sure that atleast 4 lines are below the HR
    \needspace{4\baselineskip}

    
        \vspace{6pt}
        \makebox[0.1\linewidth]{\smaller\hfill\tt\color{nbframe-in-prompt}In\hspace{4pt}{[}{]}:\hspace{4pt}}\\*
        \vspace{-2.65\baselineskip}
        \begin{ColorVerbatim}
            \vspace{-0.7\baselineskip}
            \begin{Verbatim}[commandchars=\\\{\}]

\end{Verbatim}

            
                \vspace{0.3\baselineskip}
            
        \end{ColorVerbatim}
    

        

        \renewcommand{\indexname}{Index}
        \printindex

    % End of document
    \end{document}


